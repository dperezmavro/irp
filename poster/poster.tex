\documentclass[30pt, a1paper, portrait, margin=0mm, innermargin=15mm,blockverticalspace=15mm, colspace=15mm, subcolspace=8mm]{tikzposter} 
	\usepackage{natbib}
	\usepackage{hyperref}

    \title{Attacking Microcontrollers} 
    \institute{Schoold of Electronics and Computer Science, University of Southampton}
    \author{Author: Dionisio Perez-Mavrogenis (dpm3g10)\\
			Supervisor: Klaus-Peter Zauner (kpz)}
    \usetheme{Board} %Board is also cool

    
\definebackgroundstyle{die_image}{
\includegraphics[height=\paperheight, width=\paperwidth]{opt.jpg}
}
\usebackgroundstyle{Rays}    
    
\begin{document}

	\maketitle     
    \begin{columns} % See Section 4.4
        \column{0.5} % See Section 4.4
            \block{Microcontroller Introduction}{
                Microcontrollers can be found anywhere, from your cars stereo to missile launch panels. Mictrocontrollers are the CPU of a small embedded system and are usually cheap (around \pounds 2) and widely available for hobbyists to play or companies to use in their products. 
                
                As microcontrollers are used in serious applicationsas well, they often come with crypto-engines (AES, DES and RSA are common) and hold all sorts of information like private crypto-keys for authentication or propietary algorithm implementations in the firmware or hardware, interesting all sorts of people into the contents of a microcontroller.
                
                \textbf{insert cool graphic here}
            }
            \block{Packaging and De-packaging}{
				Typically microcontrollers are too small and fragile to use as they are fabricated (with fabrication lengths shrank to micro-meters) and so they are packaged\citep{hwre}. Packacking material ranges depending on the microcontroller and its intended use,  but is usually hard epoxy resin \citep{sergei:thesis} \citep{hwre}. The packaging tries to protect the microcontroller from its external environment (humidity, radiation, temperature, crashes etc.) and also from prying eyes. Military-grade chips come with a lot of additional circuitry on the packaging whose responsibility is to detect tampering and respond in a suitable manner (even destroy itself!).
                
               
                
                depackaging ways, acids and stuff
            }
        \column{0.5}
            \block{Sample Attack}{
                provide atmega644 characteristics.
                set attack scenario, type, setup and exact details
            }
            \block{Evaluation}{
                Shit be broken, yo.
                \innerblock{qwgfwg}{qgwwg}
            }
        \note[rotate=15]{
			Very sexy, this is.        
        } % See Section 4.3
    \end{columns}
    
\bibliographystyle{plain}
    
    \block[roundedcorners=65]{}{{\footnotesize \bibliography{irp_report}}}
\end{document}