\section{Discussion}
\label{sec:conclusion}
In this report we reviewed the current attacks against MCUs and the defences developed in response. We also reviewed the relevant architectural information of the ATmega644 and then proposed a possible non-invasive clock-glitch attack on the ATmega644 board by putting together information from research done on systems with almost identical specifications. Although the exact details for performing a successful glitch, like the glitch period, would need to be found by experimentation on the target board \citep{glitches_paper}, once found they would work consistently \citep{glitches_paper} and from there the glitching results would need interpretation by the attacker in order to achieve their goal. Although conceptually simple in terms of implementation and theoretic foundation \citep{sergei:thesis} \citep{glitches_paper}, interpretation of the results is itself a challenging task \citep{glitches_paper} because of the architecture of the MCU. Since the modified Harvard architecture of the AVRs accesses both memory and instructions at the same time, the glitch will have an effect on multiple pipeline stages and its effects are influenced by the particular instructions being handled at the time and hence creating an exploitable glitch is more involved than commonly perceived \citep{glitches_paper}. 

As in traditional computer systems \emph{Security is Hard} for MCUs as well. It is difficult for an individual to completely asses the security of their board with companies making exaggerated claims about the security of their products \citep{sergei:thesis}, keeping information hidden and not always being rigorous with their analysis \citep{sergei:thesis}. Individuals can build their home probing station for probing attacks for a few thousand dollars \citep{sergei:thesis} \citep{low_cost_probing} and attempt to break their system themselves or can hire firms that do this professionally (Riscure \citep{website:riscure}, {IC-Crack} \citep{atmel_mcu_crack}) for vulnerability testing, but its security status will be assessed based on what current knowledge. Even if the progress of technology and the shrinking size of ICs make the process of micro-probing harder, but not impossible \citep{sergei:thesis} \citep{gutman:memory_remanence}, it would be reasonable to expect attackers to go after semi or non-invasive attacks, side-channels, implementation and protocol failures or bugs in very intelligent ways. The technological progress and the arms race between attackers and defenders means that each other will make their opponent more sophisticated. The majority of researchers in the MCU and smart-card field seem to agree that no system is unbreakable and one can only harden their system enough to make the effort of breaking it unbearable to those they wish to protect against \citep{anderson:cautionary_note} \cite{sergei:thesis}, i.e. make their product uninteresting to attack in terms of money and time. A complete security assessment in order to harden the device would mean that all possible threat scenarios should be considered and the security limitations of the device made perfectly clear \citep{kocher:DPA}. It is a well-established notion in the security industry that security does not \emph{add} anything to a product but is rather a huge expense to avoid a \emph{potentially} bad situation, with the effect that companies tend to not place as much focus on securing their products. Securing a product means longer release cycles, lengthier and more expensive testing, manufacturing and research phases \citep{kocher:DPA} and usually one has to choose between usability and security.