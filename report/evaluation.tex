\section{Evaluation}
\label{sec:conclusion}
\red{ mention legal issues of security through obscurity : HWRevEng}
\red{mention why security is complicated}

No system is unbreakable and one can only harden their system enough to make the effort of breaking it unbearable to those they wish to protect against\citep{anderson:cautionary_note}\cite{sergei:thesis}.

security is hard to do because one must carefully analyse and model all threats \citep{kocher:DPA}
	\subsection{Attacks and Solutions overview}
	present how expensive it would be for someone to put together a probing station and in general give estimations of cost  and skillset required for various attack categories
	\subsection{Conclusions}

security/shielding adds to a device's cost and size \citep{kocher:DPA} and there is a tradeoff \citep{sergei:thesis}


companies tend to make security claims for their products that do not stand or to market their protection as sufficient\citep{sergei:thesis}. atmega are popular and represent a typical low-cost microcontroller \citep{glitches_paper}. 

who needs secure chips ? car industry, service providers, manufacturers of various devices, banking industry and the military \url{http://www.cl.cam.ac.uk/~sps32/ECRYPT2011_1.pdf}

Mention that authors claim is not so hard \citep{sergei:thesis} but injecting an exploitable usefull fault is. Balasch et al. \citep{glitches_paper} said that what makes it had on AVR mcus to put faults is that the fault will affect both pipeline stages at the same time and found that the effects are dependent on what is currently executing/being fetched/registers and values involved. Invalid opcode



\citep{low_cost_probing} estimate around \$2000 for optical microscope + micropositioners + moving base + electronics , \citep{sergei:thesis} estimates about the same. 

\red{mention service that breaks chips }