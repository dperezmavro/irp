\section{Introduction}
	This paper will present an overview of the current attack methods for tampering with MCUs (MicroController Unit) in order to obtain access to IP (Intellectual Property), where IP refers to the firmware of the MCU and information that could be obtained from that, i.e. implementation secrets or proprietary algorithms that help a manufacturer achieve higher performance over their competitors. Firmware tampering (or theft) detection and prevention of a MCU is a popular problem with active research, as successfully addressing it would benefit the government and military, the car industry and various service providers \citep{sergei:thesis}, to name a few.
	
	A distinction between ordinary and MCUs engineered to withstand certain attack types should be made\citep{sergei:thesis} and, due to the sophistication of the protective mechanisms and the attacks, a broad classification of attackers into three groups \cite{anderson:cautionary_note}, namely :
		\begin{itemize}
			\item \textbf{Class I} Clever and curious people with a very limited budget, with some degree of knowledge and no time restrictions.\\
			\item \textbf{Class II} Professionals skilled on electronics with access to specialised equipment and resources. Their time allowance and funding might be limited.\\
			\item \textbf{Class III} Organisations with access to MCU manufacturing equipment. Their funding is usually unconstrained and their time schedule might be tight.
		\end{itemize}
	
	Firmware tampering has a number of consequences, including attackers downloading the code from a MCU and flashing it onto a MCU that they sell, effectively avoiding development and testing costs but still offering the same product as other manufacturers \cite{tech:aes_bls}, theft of private cryptographic keys or other industrial secrets and perhaps theft of service \citep{sergei:thesis}. A less obvious, but perhaps more important, case is the case of back-dooring\footnote{The act of adding code to a system without the user's knowledge or approval, usually to accomplish nefarious tasks.} a MCU by re-flashing on it a modified version of the firmware with code added by the attacker in order to accomplish their malicious intents, a very realistic scenario for governments that order electronic equipment from other countries, effectively impacting a nation by compromising secrets or activities of the government .

	It is essential for an attacker to be familiar with the technical features and architecture of the board they are trying to attack (Sec.~\ref{sec:atmega_overview}) and it is equally important to understand the current attack mechanisms (Sec.~\ref{sec:curr_attacks}) and the protective mechanisms developed in response and which particular attack type they protect against (Sec.~\ref{sec:defences}), if one wants to be effective on their attack. Once one is familiar with the above, they can decide how they want to implement their attack (Sec.~\ref{sec:attacking_mega}).