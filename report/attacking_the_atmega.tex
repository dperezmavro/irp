\section{Attacking The ATmega}
\label{sec:attacking_mega}

The ATmega is a typical MCU that is susceptible to a successful attack by the \textbf{Home-Hacker} class of attackers due to its weak protective mechanisms. While there is a way to deliver firmware updates to the MCU securely\citep{tech:aes_bls}, keeping the firmware secure on the device is impossible.

companies tend to make security claims for their products that do not stand or to market their protection as sufficient\citep{sergei:thesis}.

The motivation behind attacking the ATmega 688/1284 is their popularity and to prove how easy it is

\red{mention service that breaks chips. mention ChipWhisperer paper. perhaps mention pipeline/give examples of clock glitch. mention that no paper found targeting these devices. \citep{in_depth_glitch} this guy breaks and \citep{sergei:theis} and website of chinese dude that break atmels.}

	\subsection{Motivation}
	
	\subsection{Attack Overview}
		* added cost (in terms of \$\$, extra hardware and software implementation penalties/overhead)