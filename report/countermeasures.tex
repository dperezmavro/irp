\section{Countermeasures to known attacks}
\label{sec:defenses}

As demonstrated in the previous section a manufacturer has to guard against a multitude of attacks. For designing effective defensive mechanisms one has to enumerate the likely attack scenarios and methods, type of attacker they will be facing and decide the type and extend of confidentiality they would like to provide\citep{sergei:thesis}.

A common first line of defense that is often encountered in attempts to secure systems, with questionable effectiveness, is security by obscurity and in the case of MCUs can be achieved in a number of different ways. As a starting point, the manufacturers can simply avoid printing their logos or model numbers on parts they produce, as this will make information gathering for he chip trickier. Some vendors take this even further by printing on their products identifiers of more secure or expensive products \footnote{Related legal issues discussed in Sec.~\ref{sec:conclussion}\red{hardware reveng legal issues}} or even try to make their MCUs look like ASICs \red{abbreviation?} in order to scare away potential attackers by making their product appear more secure\citep{sergei:thesis}. A further step in making the MCU harder to analyze is hardware obfuscation, where a group of gates commonly placed in well known structures for performing a given task (for example DES circuitry) is intentionally made more complicated and physically placed in a different locations or additional circuitry is added in order to thwart analysis\citep{anderson:cautionary_note}. Vendors also try to make information on their products hard to find by selling only to selected partners or under non-disclosure agreements\citep{sergei:thesis}.

Security through obscurity has received lots of criticism, and rightfully so, as it will do little to stop experienced or determined attackers. Despite its financial appeal a more effective approach is security in depth, a model in which information is protected by a number of (potentially) different defense mechanisms and in this scenario could be employed at different levels of the chip, from its external potting (or encasing) to its layered architecture. Given the attack categories presented in Sec.~\ref{sec:curr_attacks}, it would be useful to broadly categorize denfences in a similar fashion.

Non-invasive attacks are the least sophisticated type of attack but protection against them is a relatively laborious process\citep{anderson:cautionary_note}. To start with, one has to mask all the side-channel leakage and this means mask all relationships between data input and power consumption, thermal and electromagnetic radiation and timing relationships\citep{kocher:DPA}\cite{sergei:thesis}. Given the software release model and the power of MCUs it's not unreasonable to strive for efficiency, both in terms of power and components. Bytecode optimizations performed by compilers and the hardware architecture of the instruction pipeline of an MCU might introduce leakage potential \citep{kocher:DPA}\citep{sergei:thesis} by how branches are taken, instructions pre-fetched or not take constant time for operations. Even if constant time is taken for correct and erroneous keys such that timing relationships are masked, intermediate result storage is still a problem due to the fact that a 0 and a 1 consume different power when handled (due to the physical nature of CMOS transistors) and hence the power consumed in a given operation is proportional to the amount of 1 bits, a concept known as "Humming weight"\citep{website:riscure}\citep{kocher:DPA}\red{maybe read this about CPA as well https://www.iacr.org/archive/ches2004/31560016/31560016.pdf }. Electromagnetic or thermal emission detection can be avoided by packaging that is appropriately shielded\red{reference needed}. \red{also destroying circuitry and lock bits}

also have self health tests to detect alteration\citep{anderson:tamper_resistance}

Protection against invasive and non-invasive attacks is a lot harder because these attacks involve a range of chip modifications (from decapsulation to modification of the die), requiring varied anti-tampering mechanisms; we will proceed to give a description in the order in which decapsulation occurs. Thwarting decapsulation can get very creative as decapsulation is dependent on the material from which the chip packaging is made from (varied from pure plastic to . tries In order to protect from semi-invasive attacks (such as exposure to radiation or optical inspection
\subsection{Physical Protection}
asic glue logic design, protective metal-mesh layer, shrink things, mesh of wires, encase in epoxy, make layers destroy each other if removed. doping can also reduce light penetration to reduce imaging attacks 
\subsubsection{Protection circuits}
radiation/temperature/voltage/frequency detector circuits that cause reset on abnormality detection (instability reference \citep{anderson:cautionary_note}). add transistors on top to hide true signal (provide reference)
 keep keys in own, self powered module.
\subsection{Side-channel Protection}
decrease signal/noise ratio (either by introducing noise or making the signal smaller), constant time/power operatations, insert random delays\citep{sergei:thesis}, \citep{kocher:DPA} and shielding the device.

Planarization defeats optical inspection with microscope (source: Introduction to hardware security and trust, sergei skorobogatov paper).\\
	\begin{itemize}
	\item overview of most popular techniques \\
	\item benefits and how they improve the situation/approach the problem
	\item added cost for this investment (in terms of hardware and money, transparency to the developers, runtime overhead etc)\\
	\end{itemize}
	
	perhaps review some popular secure chips ?? IBM 4758 is a secure device \footnote{\href{http://www.cl.cam.ac.uk/~rnc1/descrack/ibm4758.html}{http://www.cl.cam.ac.uk/~rnc1/descrack/ibm4758.html}}, some Dallas chips and perhaps more.
	
	thermally enhanced packaging : makes difficult to microprobe chips as they need to be cooled or they'll fry [slides of hardware reveng module] and careful decaping needed
	
	ceramic packaging (Al2O3 is common, often mixed with SiO2, AlN and BeO) is extremely chemical resistant and seals very good, but can crack
	
	potting : protects board from environment[water/dirt] but might also make tampering harder and may need to be removed
	conformal coating : same deal as potting, but it's a thin layer that doesn't fill the enclosure but just covers the top.
	underfill : provides stability but gets in the way as well(makes desoldering harder).