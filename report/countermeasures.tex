\section{Countermeasures to known attacks}
\label{sec:defenses}
\subsection{Physical Protection}
protective metal layer, shrink things, mesh of wires, encase in epoxy, make layers destroy each other if removed
\subsubsection{Protection circuits}
radiation/temperature/voltage/frequency detector circuits that cause reset on abnormality detection (instability reference \citep{anderson:cautionary_note}). add transistors on top to hide true signal (provide reference)
 keep keys in own, self powered module.
\subsection{Side-channel Protection}
decrease signal/noise ratio (either by introducing noise or making the signal smaller), constant time/power operatations, insert random delays\citep{sergei:thesis}, \citep{kocher:DPA} and shielding the device.

Planarization defeats optical inspection with microscope (source: Introduction to hardware security and trust, sergei skorobogatov paper).\\
	\begin{itemize}
	\item overview of most popular techniques \\
	\item benefits and how they improve the situation/approach the problem
	\item added cost for this investment (in terms of hardware and money, transparency to the developers, runtime overhead etc)\\
	\end{itemize}
	
	perhaps review some popular secure chips ?? IBM 4758 is a secure device \footnote{\href{http://www.cl.cam.ac.uk/~rnc1/descrack/ibm4758.html}{http://www.cl.cam.ac.uk/~rnc1/descrack/ibm4758.html}}, some Dallas chips and perhaps more.