\section{Countermeasures to known attacks}
\label{sec:defenses}

As demonstrated in the previous section a manufacturer has to guard against a multitude of attacks. For designing effective defensive mechanisms one has to enumerate the likely attack scenarios and methods, type of attacker they will be facing and decide the type and extend of confidentiality they would like to provide\citep{sergei:thesis}\citep{kocher:DPA}.

A common first line of defence that is often encountered in attempts to secure systems, with questionable effectiveness, is security by obscurity and in the case of MCUs can be achieved in a number of different ways. As a starting point, the manufacturers can simply avoid printing their logos or model numbers on parts they produce, as this will make information gathering for he chip trickier. Some vendors take this even further by printing on their products identifiers of more secure or expensive products \footnote{Related legal issues discussed in Sec.~\ref{sec:conclusion}\red{hardware reveng legal issues}} or even try to make their MCUs look like ASICs \footnote{\red{abbreviation?}} in order to scare away potential attackers by making their product appear more secure\citep{sergei:thesis}. A further step in making the MCU harder to analyse is hardware obfuscation, where a group of gates commonly placed in well known structures for performing a given task (for example DES circuitry) is intentionally made more complicated and physically placed in a different locations or additional circuitry is added in order to thwart analysis\citep{anderson:cautionary_note}. Vendors also try to make information on their products hard to find by selling only to selected partners or under non-disclosure agreements\citep{sergei:thesis}.

Security through obscurity has received lots of criticism, and rightfully so, as it will do little to stop experienced or determined attackers. Despite its financial appeal a more effective approach is security in depth, a model in which information is protected by a number of (potentially) different defence mechanisms and in this scenario could be employed at different levels of the chip, from its external potting (or encasing) to its layered architecture. Given the attack categories presented in Sec.~\ref{sec:curr_attacks}, it would be useful to broadly categorize defences in a similar fashion.

Non-invasive attacks are the least sophisticated type of attack but protection against them is a relatively laborious process\citep{anderson:cautionary_note}. A first step to prevent unauthorized access to the firmware is using lock bits and fuses\citep{atmega_manual}\citep{tech:avrfreaks} and some manufacturers (or developers) go as far as physically destroying reading pins or cutting out testing circuitry\citep{sergei:thesis}. Further protection is introduced by avoiding side-channel leakage by masking all relationships between data input and power consumption, thermal and electromagnetic radiation and timing relationships\citep{kocher:DPA}\cite{sergei:thesis}. Given the software release model and the power of MCUs it's not unreasonable to strive for efficiency, both in terms of consumption and components, and bytecode optimizations performed by compilers and the hardware architecture of the instruction pipeline of an MCU might introduce leakage \citep{kocher:DPA}\citep{sergei:thesis} by how branches are performed, instructions pre-fetched or not take constant time for operations (or execution of different instructions). Even if constant time for cryptographic operations is taken for correct and erroneous keys, or random delays are introduced such that timing relationships are masked, storing intermediate results still poses a problem due to the fact that a 0 and a 1 consume different power when handled (due to the physical nature of CMOS transistors) and hence the power consumed in a given operation is proportional to the amount of 1 bits, a concept known as "Hamming weight" \citep{website:riscure}\citep{kocher:DPA}\footnote{\red{maybe read this about CPA as well https://www.iacr.org/archive/ches2004/31560016/31560016.pdf}}. More approaches to thwarting power analysis include using a lower operating voltage so that power fluctuations are less evident or is introducing noise by means of a random number generator with variable power consumption\citep{kocher:DPA}\citep{hwre}\citep{kocher:DPA}. Electromagnetic or thermal emission detection can be avoided by packaging that is appropriately shielded\citep{website:ibm_secure}\citep{kocher:DPA}. 

Protection against invasive and semi-invasive attacks is a lot harder because these attacks involve a range of chip modifications (from decapsulation to modification of the die), requiring varied anti-tampering mechanisms; we will proceed to give a description in the order in which decapsulation occurs. Thwarting decapsulation can get very creative as decapsulation techniques depend on the material from which the chip packaging is made of (varied from pure plastic to ceramic or metal) and its physical construction, with techniques ranging from using a sharp object to pull the top off \citep{sergei:thesis} to using acids and other chemicals \citep{hwre}\citep{sergei:thesis} to eat the top layer away and expose the die surface and manufacturers trying to make the decapsulation process as hard as possible. Common techniques involve sealing the die in conductive packaging or making the packaging from conductive and very hard epoxy resin such that packaging removal will result in power supply loss and a response from intrusion detection circuitry from the chip; common responses are erasing sensitive keys from internal battery backed SRAM\citep{hwre}, resetting of the device \citep{sergei:thesis} or, for military grade equipment, have reactive chemicals or little charges embedded in the packaging that would respond to stimuli such as other chemicals or small currents (as a result of optical imaging) in a very violent manner and destroy the chip completely. 

The next level of protection came when radiation attacks became known\citep{kocher:DPA}, where attackers used to override fuses by exposing them to UV light. The defensive response to that was the addition of opaque metal rectangles on top of critical components, or the entire die, in order to shield them from radiation and scattering additional security fuses, which were usually hidden near critical memory areas like the IVT in order to damage other components as well when tampered\citep{sergei:thesis}\citep{hwre}. A further step to protect the top of the device was to place conductive wire mesh layer(s) over the chip area before it is embedded in epoxy , like in the IBM $\mu$ABYSS \citep{website:ibm_secure} and the Atmel ATSHA204\citep{hwre}, in an attempt to detect tampering and erase the keys from SRAM if the power is disrupted. These mechanisms attempt to thwart micro-probing and modification attacks (both memory contents and circuitry modification)  but also prevent visual inspection and analysis of the die\citep{hwre}, making the localisation of critical components harder and are commonly found in smart-cards\citep{sergei:thesis} and SIM cards as well\citep{hwre}. Another approach to preventing visual inspection and IR imaging (without de-layering the die) is the doping of semiconductor materials in order to reduce light penetration\citep{sergei:thesis}, as well as chemical or mechanical planarization\footnote{planarization techniques here \url{http://www.stanford.edu/class/ee311/NOTES/Deposition_Planarization.pdf}} of the layers of the MCU\citep{sergei:thesis}.



also have self health tests to detect chip modification \citep{anderson:tamper_resistance}

make wires into a maze to make human analysis more difficult \citep{hwre}.

In order to protect from semi-invasive attacks (such as exposure to radiation or optical inspection) doping can also reduce light penetration to reduce imaging attacks \citep{sergei:thesis})
tilt sensors \citep{website:ibm_secure}

optical sensors (\citep{sergei:thesis}, IBM reference, \citep{hwre}
glitch sensors (power/radiation/temperature/clock) - use of multiple clocks, \red{instability reference \citep{anderson:cautionary_note}}
tilt sensors 
Planarization defeats optical inspection with microscope \citep{sergei:thesis}.\\
things also shrink





not perfect : give example for failure of oprical sensors
Scatter unshielded phototransistors around
Trigger when illuminated
May not detect laser glitching in a dark room

\subsection{Physical Protection}
asic glue logic design, protective metal-mesh layer, shrink things, mesh of wires, encase in epoxy, make layers destroy each other if removed. 
\subsubsection{Protection circuits}
 keep keys in own, self powered module.
	
	perhaps review some popular secure chips ?? IBM 4758 is a secure device \footnote{\href{http://www.cl.cam.ac.uk/~rnc1/descrack/ibm4758.html}{http://www.cl.cam.ac.uk/~rnc1/descrack/ibm4758.html}}, some Dallas chips and perhaps more.
	
	thermally enhanced packaging : makes difficult to microprobe chips as they need to be cooled or they'll fry [slides of hardware reveng module] and careful decaping needed
	
	ceramic packaging (Al2O3 is common, often mixed with SiO2, AlN and BeO) is extremely chemical resistant and seals very good, but can crack
	
	potting : protects board from environment[water/dirt] but might also make tampering harder and may need to be removed
	conformal coating : same deal as potting, but it's a thin layer that doesn't fill the enclosure but just covers the top.
	underfill : provides stability but gets in the way as well(makes desoldering harder).


Tamper responses
● Freeze (gate clock)
● Reset
● Self-destruct (erase firmware/data/keys)

● Flash erase
– Can be prevented
– Laser/FIB/etch out charge pump caps
– Cut/short write enable lines, HV outputs, etc
– No HV = no writes
● Zeroize battery-backed SRAM
– Much harder to prevent