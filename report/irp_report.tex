%\documentclass[10pt,journal,compsoc,a4paper]{IEEEtran}
\documentclass[10pt,a4paper,twocolumn]{article}

%for links
%\usepackage[hidelinks]{hyperref}
\usepackage{hyperref}

%for bilbiography
\usepackage{natbib}

%for images
\usepackage{graphicx}
\usepackage{caption}
\usepackage{subcaption}

\author{Dionisio Perez-Mavrogenis}
\title{Firmware Protection and Attacks Against the ATmega Microcontrollers}
\date{\today}

\begin{document}
	\maketitle
	
\section*{\emph{Abstract}}
	\textbf{\emph{The most abstract abstract of some abstracts, full of abstractions.}}
	

\section{Introduction}
	This paper will present an overview of the current attacks and methods of tampering with IP\footnote{IP = Intellectual Property} in MCUs\footnote{MCU = MicroController Unit}. IP refers to the firmware running on the MCU and information that could be obtained from that, i.e. implementation secrets or proprietary algorithms that help a manufacturer achieve higher performance over their competitors. Firmware tampering (or theft) detection and prevention of a MCU is a popular problem with active research, as successfully addressing it would benefit a lot of parties (including the government and the military).
	
	A distinction between ordinary and secure MCUs should be made\citep{sergei:thesis} and, due to the sophistication of the protective mechanisms and the attacks, a broad classification of attackers as\cite{anderson:cautionary_note}:
		\begin{itemize}
			\item \textbf{Home Hackers} Clever and curious people with a very limited budget, perhaps with some degree of knowledge and no malicious intent, but with no time-limit.\\
			\item \textbf{Semi-Professional Crackers} Professionals skilled on electronics with access to specialised equipment and resources. Their funding might be limited, malicious intent is unclear and they might be constrained in their time allowance.\\
			\item \textbf{Funded Organisations} Organisations with access to MCU manufacturing equipment. Their funding is usually unconstrained, there usually exists malicious intent and their time schedule is usually tight.
		\end{itemize}
	
	Firmware tampering has a number of consequences, the most obvious being an attacker downloading the code from a MCU and flashing it onto a MCU that they sell, effectively avoiding development and testing costs but still offering the same product as other manufacturers\cite{tech:aes_bls}. A less obvious, but perhaps more important, case is the case of back-dooring\footnote{The act of adding code to a system without the user's knowledge or approval, usually to accomplish nefarious tasks.} a MCU by re-flashing on it a modified version of the firmware with coded added by the attacker in order to accomplish their malicious intents, which could have disastrous consequences if these MCUs were used for military or other sensitive operations. 

	\subsection{Objectives}
	The aims of this paper are to review the possible attacks against the ATmega series of MCUs and provide possible countermeasures or possible methods of hardening a system. 
	
	Section \ref{sec:atmega_overview} will provide an overview of the ATmega series of AVRs and explain the most important hardware architecture aspects and protection mechanisms they offer. 
	
	Section \ref{sec:curr_attacks} will give a (brief) overview of the current attack techniques used to override currently implemented protection mechanisms, an overview of whom is given in Section \ref{sec:defenses}. 
	
	In section \ref{sec:attacking_mega} the current attacks will be related to the ATmega, by presenting the attack vectors in more detail as well as providing references to relevant work. Section \ref{sec:conclusion} will conclude the paper with a discussion on the usefulness of hardening the ATmega, contrasting that with using a MCU that is designed to be secure.
\section{The AVR MCU Series}
\label{sec:atmega_overview}

	In this paper we will focus on the ATmega644 board, of the Atmel AVR line of MCUs, as it is a widely available and popular low-cost microcontroller \citep{glitches_paper} . The AVR series is an enhanced-RISC architecture 8-bit MCU family that consists of the ATtiny, ATmega and ATxmega sub-categories,  32-bit AVRs and application specific FPGAs\cite{book:practical_avr}. The models have varying degrees of hardware capabilities and large operating voltage windows in order to accommodate demand and integrate well with peripherals\footnote{info:\href{https://www.newbiehack.com/MicrocontrollersAlternativePowerSources.aspx}{https://www.newbiehack.com}}\footnote{info:\href{http://www.atmel.com/v2PFResults.aspx}{http://www.atmel.com/v2PFResults.aspx}}. Developing software for an AVR is easy as the AVRs benefit from the  \texttt{avr-libc} high-performance library, the \texttt{avr-gcc} and \texttt{avr-gdb} compiler and debugger(both based on very popular and high quality GNU software tools), the \texttt{avrdude} programming software(or Atmel's \texttt{AVRStudio}) and \texttt{Simulavr} simulator software. Additionally, Atmel provides proprietary APIs for interacting with the AVR and the developers can choose from a wide variety of programmer units available for working with the AVRs\cite{book:practical_avr}.
	
	\subsection{ATMega Architecture and Features}				

	This paper will focus on the ATmega644, an enhanced-RISC Harvard architecture 8-bit CPU with a two stage pipeline and a total of 131 instructions. Fig.~\ref{fig:architectures} shows the conceptual difference between a Von Neuman and strict Harvard architecture, where the key distinction lies in the separation of application code and program data into different memory sections (Harvard) and tasking the CPU with distinguishing between code and data present in the same memory region (Von Neuman). The 644 implements a modified Harvard architecture for both power and computational efficiency, designed to access multiple memory locations simultaneously thus being able to execute an instruction per cycle, as shown in Fig.~\ref{fig:pipeline}. Their speed grades are  0-10 MHz for 2.7 V - 5.5 V and 0 - 20 MHz for 4.5 V - 5.5 V \citep{atmega_manual}.

	The 644 is equipped with 2 Kb of EEPROM, 64 Kb of flash memory, 4 Kb of SRAM, and a large number of general purpose (GP) and I/O registers and all memory (including memory mapped I/O images) is linear, i.e. it follows the flat memory model. The flash memory is separated into the bootloader and application code section and the boundary between the two sections, as well as the page size, can be configured by programming the appropriate fuses. Both sections hold code, however code residing in the bootloader section can execute the \texttt{SPM}\footnote{\texttt{SPM} = Store Program Code, assembly instruction for the AVR.} instruction which allows the bootloader code to write to \textit{any} section in the flash memory and hence possibly modify itself, facilitating purposes like firmware upgrades. The bootloader code can be triggered by a direct jump from the application section or by programming the reset vector via the reset fuse to point to the appropriate section of the bootloader code. The EEPROM is memory for data that needs to persist between reboots of the MCU and hence it is (widely) used to hold configuration variables and other non-temporary data the application code (or the bootloader) may need, having an average lifespan of 100,000 write cycles per page. The SRAM is volatile storage and is used as the stack and heap for the firmware and for storing the Register File, i.e. the 32 GP registers, I/O and Extended I/O Memory. The reserved register locations exist in order to support the use of peripheral units as well as hold program status information (e.g. the Stack Pointer can be found in one of the GP registers). Fig.~\ref{fig:stack} gives an overview of the SRAM hierarchy for the ATmega644.
	
	\begin{figure}
		\center
		\includegraphics[scale=0.7]{img/stack.png}
		\caption{\footnotesize SRAM layout for ATmega644 (source: \protect\citep{atmega_manual}).}
		\label{fig:stack}		
	\end{figure}

	\begin{figure}
		\center
		\includegraphics[scale=0.7]{img/pipeline.png}
		\caption{\footnotesize 2-stage pipeline of the ATmega644 (source: \protect\citep{atmega_manual}).}
		\label{fig:pipeline}		
	\end{figure}
	
\begin{figure*}
	\begin{subfigure}{0.5\textwidth}
		\center
		\includegraphics[scale=0.5]{img/von_neuman_arch.jpg}
		\caption{\footnotesize Typical Von Neuman architecture.}
		\label{fig:VN_arch}
	\end{subfigure} 
	~
	\begin{subfigure}{0.5\textwidth}
		\center
		\includegraphics[scale=0.5]{img/harvard_arch.jpeg}
		\caption{\footnotesize Strict Harvard architecture.}
		\label{fig:H_arch}
	\end{subfigure}
	\caption{\footnotesize A comparison of different machine architectures (source: \protect\citep{website:mcu_primer}).}
	\label{fig:architectures}
\end{figure*}	
	
	\subsection{ATMega Security Features}
	
	The AVR ATmega644 is not meant to be a secure hardware module but offers firmware access control by using six lock bits responsible for controlling access to the board's memory and prevent reading or modifying the memory (e.g. prevent code executing from the bootloader section to read/write the application code section using \texttt{SPM}). This access control is not permanent, as that would limit the usefulness of the MCU and therefore one has the option to reset the lock bits (i.e. having no protection scheme enabled) by issuing a Chip Erase command, which has the effect of completely erasing the Flash, EEPROM and then the lock bits \citep{atmega_manual}. Chip erasing is performed with the sequence of events as presented and this is important, as one does not want to remove the access protection before removing all data and hence the lock bits are set to 1 only after the whole program memory has been erased. Even though the flash memory has an average lifespan of 10,000 write cycles, as well as programming being a relatively lengthy operation, this approach tries to preserve the intellectual property on the board rather than the board itself.
	
	\begin{table}
		\caption{\footnotesize Security lock bits offered by the ATmega644. BLB stands for Boot Lock Bit and LB for Lock Bit. (source: \protect\citep{tech:avrfreaks} \citep{atmega_manual})}
		\label{table:lock_bits}
		\center
		\begin{tabular}{| c | c | c | c |}
			\hline
			\textbf{Lock Bit Byte} & \textbf{Bit Number} & \textbf{Default}\\
			\hline \hline
			BLB12 & 5 & 1\\
			BLB11 & 4 & 1\\
			BLB02 & 3 & 1\\
			BLB01 & 2 & 1\\
			LB2 & 2 & 1 \\
			LB1 & 1 & 1 \\
			\hline
		\end{tabular}
		
	\end{table}
	
Table~\ref{table:lock_bits} presents an outline of the available Lock bits provided by the ATmega series. The functionality of the BLB1 group is to control access and modification of the bootloader section, group BLB0 bits control access to the application code section and group LB bits are responsible for controlling modifications on the EEPROM and Flash. A detailed explanation of their functionality and how to use them is given in the manuals \citep{atmega_manual} \citep{tech:avrfreaks}.
	

\section{Attacks on Hardware}
\label{sec:curr_attacks}
A distinction between \emph{passive} and \emph{active} attacks should be made. In the former the attacker simply monitors the chip's normal operation and tries to infer the input-output mapping whereas in the latter case the attacker actively manipulates either the chip or its operating environment with the aim of obtaining insight on the chips inner workings. 

Attacks on MCUs may attempt to recover a number of artefacts, including cryptographic keys the firmware and do not need to necessarily attack the hardware itself but can exploit flaws in algorithmic design and implementation and protocol failures or inter-component communication patterns\citep{anderson:cautionary_note}\citep{kocher:DPA}, obtain information by corrupting the memory or exploiting memory remanence\citep{sergei:thesis}\citep{gutman:memory_remanence}.

	\subsection{Non-Invasive Attacks}
	Non-invasive attacks are attacks which require no depackaging or special preparation of the chip and hence attacks under this category leave little tamper evidence behind. These attacks might be very time consuming and are not guaranteed to be successful, but are very easy and cheap to replicate once found. Furthermore, non-invasive attacks could target badly implemented communication or security protocols in order to bypass security restrictions.
	
	\subsubsection{Power Analysis}
	\label{subsubsec:power_analysis}
	Different instructions executing on a CPU require different amounts of power, and hence one can infer which instruction is executing on a CPU by analysing a power trace generated by the MCU. These attacks are easy and relatively inexpensive to perform as they only require widely available tools.
	
	Simple Power Analysis(SPA) involves direct observation of the MCU when it performs cryptographic operations and can leak information about both the keys and the cryptographic operations themselves (i.e. nature or structure of the algorithm). 

	Differential Power Analysis(DPA) extracts sensitive information by using statistical techniques on very large traces. The techniques involves obtaining power traces of known cipher-texts(but not necessarily knowing the cipher-texts) and individual bits of the key are recovered by analysing the differences in power consumption\citep{kocher:DPA}.
	
	One can generally avoid noise in their power measurements by sampling the voltage (usually) on the ground line.
	\subsubsection{Glitch Attacks}
	\emph{Power glitches} and \emph{clock glitches} aim to make the CPU skip or execute incorrect instructions by applying transients. This attack can target in individual components of an MCU and a systematic search can deduce which components are affected by a given glitch sequence.
	
	Clock glitches involve increasing the clock signal frequency so that some flip flops sample their input before being updated and hence report an incorrect value. Clock glitches are mainly aimed against software-based protection mechanisms, affecting CPU operation by supplying the CPU with incorrect data.
	
	Power glitches work by supplying either too much power or too little, shifting transistors' threshold and causing flip-flops to read their state incorrectly. Power glitches need to be carefully synchronised with the internal clock and prolonged attacks might damage the board.
	
	Glitch attacks are especially dangerous as they may abuse the program counter in order to map out the memory, which is linear as described in \ref{sec:atmega_overview}.
	
	\subsubsection{Data Remanence}
	Prolonged exposure of SRAM cells to the same values can make the cells 'remember' their state, due to material properties and stress\citep{gutman:memory_remanence}. If for example a start-up routine always writes security keys to the same memory region, after some time they key will be recoverable by looking at the physical state of the memory. Furthermore, data can be recovered from SRAM  by cooling it down for a period of time after power is removed.
	
	EEPROM suffers as well, but to a lesser extent, as material-wise one can only tell virgin-cells from used cells\citep{sergei:thesis}. 
	
	\subsubsection{Timing Attacks}
	Timing attacks exploit the software implementation of cryptographic algorithms. Compiler optimisations (avoiding unnecessary branches, register and cache usage) and other implementation choices make the execution time of an algorithm dependent on the input and the secret key, rather being fixed for any input. For example, when input is compared byte-wise with a key and rejected when the first non-matching byte is found, rather than first consuming the whole input string.
	
	Different instructions take different time to execute(e.g. \texttt{MOV eax,[eax]} is considerably slower than \texttt{INC eax}) and thus one could collect timing information for various input messages and systematically deduce the correct key. 
	
	If timing information is correlated with power analysis then defences such as constant instruction execution time could be defeated. One might use \texttt{NOP}s in the case of a wrong key in order for rejection and confirmation responses to have constant execution time but \texttt{NOP} consumes substantially less power than \texttt{INC eax} and correlating timing and power consumption information would reveal this.

	\subsection{Semi-Invasive Attacks}
	Semi-invasive attacks require depackaging of the chip but do not destroy the passivation layer as no electrical contact with the chip is needed. Semi-invasive attacks can be automated and can yield results faster and cheaper than invasive attacks.
	
	\subsubsection{UV Light Exposure}
	Older chips and chips that are designed to withstand low-cost non-invasive (i.e. no depackaging) attacks are susceptible to having parts of the memory altered if it is exposed under UV light. Security fuses that prevent read-back of the memory could have their state reset by sufficient exposure under UV light. The attacker must locate the security fuse though, which can be very tricky. 

	\subsubsection{Backside Imaging}
	Backside imaging involves shining IR light on the rear side of the chip and imaging it from this angle, since it is a mirror-image of the front side. This is possible because, usually, light shown through the backside does not have to go though multiple layers and hence protective metal meshes(discussed in Section \ref{sec:defenses}) or normal chip layers are avoided. On some chips it is possible to extract ROM contents via this technique.
	
	\subsubsection{Photo Carrier Stimulation}
	Optical Beam Induced Current and Light Induced Voltage Alteration are the two most common techniques fot failure analysis that take advantage of the photoelectric effect. These techniques involve shining lasers on the semiconductor surface in order to alter some property; in OBIC a slight current is created and by analysing this one can deduce the device's properties (including defects and anomalies) and produce an image of the the board being scanned, while in LIVA the board is connected to constant power supply and changes in the power supply are monitored as laser is shone on the device, allowing one to deduce the device's characteristics and construct an image.
	
	Similar to Backside Imaging, lasers can be used to extract ROM contents.
	
	\subsection{Invasive Attacks}
	Invasive attacks require direct access to the board's surface and as a result destroy the packaging in the process, therefore leaving tamper evidence. Invasive attacks usually aim to understand how a MCU works and then develop cheaper attacks for that chip, as invasive attacks are laborious, require expensive equipment and highly skilled attackers.
	
	\paragraph{Exposing the chip surface} This usually involves destroying the packaging by using chemicals or drilling(or other methods). While this is a process that is not very complicated\citep{sergei:thesis}, one might have trouble finding the chemicals required. An alternative for depackaging the chip is to send it to a failure analysis lab\citep{website:hacking_the_pic}.
	
	\paragraph{Reverse engineering} both hardware and the software. Hardware reverse engineering requires using reflected light microscopes or SEM\footnote{SEM = Scanning Electron Microscope} for constructing a complete image of the surface. Layer removal might be required if deeper layers are not visible in order to have a complete view of the device. Software reverse engineering can be accomplished when one has obtained access to the memory.
	
	\paragraph{Micro-probing and Modification} In micro-probing sub-micrometer thickness probes are used to establish contact with the bus lines in order to observe and manipulate bus signals. To achieve reliable results a micro-probing workstation\footnote{all moving components should have micrometer precision} is used, consisting of a microscope, micro-positioners for the probes, a movable base and a test socket to place the chip. In order to establish contact with the bus lines the passivation layer should be removed, usually done with UV or green lasers, and access to bus lines of deeper layers can be achieved by using a FIB workstation.
	
	Modifications to the MCU's components (adding new interconnects or destroying circuits) are not always necessary, but could prove useful. For chip modifications to be successful, the attacker must be sophisticated and must have at least partially reverse-engineered the board.
	
	
\section{Countermeasures to known attacks}
\label{sec:defenses}
\subsection{Physical Protection}
protective metal layer, shrink things, mesh of wires, encase in epoxy, make layers destroy each other if removed
\subsubsection{Protection circuits}
radiation/temperature/voltage/frequency detector circuits that cause reset on abnormality detection (instability reference \citep{anderson:cautionary_note}). add transistors on top to hide true signal (provide reference)
 keep keys in own, self powered module.
\subsection{Side-channel Protection}
decrease signal/noise ratio (either by introducing noise or making the signal smaller), constant time/power operatations, insert random delays\citep{sergei:thesis}, \citep{kocher:DPA} and shielding the device.

Planarization defeats optical inspection with microscope (source: Introduction to hardware security and trust, sergei skorobogatov paper).\\
	\begin{itemize}
	\item overview of most popular techniques \\
	\item benefits and how they improve the situation/approach the problem
	\item added cost for this investment (in terms of hardware and money, transparency to the developers, runtime overhead etc)\\
	\end{itemize}
	
	perhaps review some popular secure chips ?? IBM 4758 is a secure device \footnote{\href{http://www.cl.cam.ac.uk/~rnc1/descrack/ibm4758.html}{http://www.cl.cam.ac.uk/~rnc1/descrack/ibm4758.html}}, some Dallas chips and perhaps more.
\section{Attacking The ATmega}
\label{sec:attacking_mega}

In this section we present the suggested attack that should work against the ATmega644. We believe that someone with a modest level of competence in electronics could successfully bypass the security fuse and lock-bit protections on the ATmega644 board, as researches have succeeded in bypassing the protection imposed by the AVR family in numerous ways. All researches used non-invasive attacks in order to achieve this since there is no need for more sophisticated attacks.Clock glitch attacks against AVR chips were successfully applied by \citep{glitches_paper} against an ATmega163 and by \citep{chipwhisperer} against an ATmega328P and \citep{avr_mega} managed to retrieve AES and DES cryptographic keys from an ATmega16 and ATXmega128A1. Furthermore, \citep{sergei:thesis} also points out that AVR MCUs are susceptible to glitch attacks due to the implementation of their security fuses. 

The suggested attack method will closely follow \citep{glitches_paper}, further supported by results from other research as well. We will be a Class-I attacker performing a non-invasive clock-glitch attack on an ATmega644-based product. The equipment we should have access should be found in any descent university's electronics laboratory and our information regarding the chip will come from datasheets. We believe that the attack engineered by \citep{glitches_paper} on the ATmega163 can be ported over to the ATmega644 due to their similarities. Both devices belong to the same family and by comparing their datasheets we can see that the 744 possesses all the hardware features that make the attack possible on the 163, namely it operates on an external clock signal and has a two stage pipeline. Furthermore, the two devices have an almost identical instruction set.\\

The attack should be tailored to a particular goal and hence we should distinguish the task of recovering cryptographic material from the task of making the device output its memory contents or execute code other than it was intended to, e.g. skipping instructions or executing other instructions. In both cases monitoring the device's power consumption will help us gain some insight into what the device is doing. 



might be hard to sync with internal clock \citep{sergei:thesis}\citep{avr_mega}



\red{mention service that breaks chips. mention ChipWhisperer paper. perhaps mention pipeline/give examples of clock glitch. mention that no paper found targeting these devices. \citep{glitches_paper} this guy breaks and \citep{sergei:theis} and website of chinese dude that break atmels.}
\section{Evaluation}
\label{sec:conclusion}

No system is unbreakable and one can only harden their system enough to make the effort of breaking it unbearable to those they wish to protect against\citep{anderson:cautionary_note}\cite{sergei:thesis}.

security is hard to do because one must carefully analyse and model all threats \citep{kocher:DPA}
	\subsection{Attacks and Solutions overview}
	present how expensive it would be for someone to put together a probing station and in general give estimations of cost  and skillset required for various attack categories
	\subsection{Conclusions}
	
	\bibliographystyle{plain}
	\bibliography{irp_report}	
	
\end{document}
