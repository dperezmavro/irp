\documentclass[12pt,a4paper,onecolumn]{article}
\author{Dionisio Perez-Mavrogenis}
\title{Code protection and attacks against IlMatto}

\begin{document}
\maketitle

This research project will look at the various issues concerned with protecting Intellectual Property(IP) in microcontroller units(MCU). The motivation behind this is the rise of the "Web of things", a notion that everyday devices will have microcontrollers embedded in them in order to make them more reactive and adaptive to a person's life, the increasing introduction of MCUs in teaching environments and their long history of industrial applications.\\

The research will focus on the IlMatto board, designed and used in the ECS. The reason for this is that it is an open-source hardware platform and thus information for it is more readily available, it is based on a very popular MCU and is used in the ECS. Moreover, making the research platform-specific will allow for a deeper understanding of the problems and countermeasures and provide a more engaging research topic, since the focus is not on an idealised hardware platform (however, there is little loss of generality as most boards have a similar structure and thus are affected by the same issues).\\

The aims of this research is to provide a comprehensive and coherent overview of the current attack vectors and implications in the case of a successful attacker, as well as provide some possible countermeasures to hinder IP unlawful duplication.


\end{document}